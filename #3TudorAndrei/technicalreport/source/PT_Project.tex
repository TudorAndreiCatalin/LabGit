\documentclass[14pt]{article}
\usepackage{listings}
\usepackage{color}
\usepackage{graphicx}
\usepackage{setspace}

\definecolor{dkgreen}{rgb}{0,0.6,0}
\definecolor{gray}{rgb}{0.5,0.5,0.5}
\definecolor{mauve}{rgb}{0.58,0,0.82}

\lstset{frame=tb,
  language=C,
  aboveskip=3mm,
  belowskip=3mm,
  showstringspaces=false,
  columns=flexible,
  basicstyle={\small\ttfamily},
  numbers=none,
  numberstyle=\tiny\color{gray},
  keywordstyle=\color{blue},
  commentstyle=\color{dkgreen},
  stringstyle=\color{mauve},
  breaklines=true,
  breakatwhitespace=true
  tabsize=3
}







\begin{document}
\title{\huge The task-scheduling problem}
\date{\today}
\maketitle
\begin{center}
\vspace{30 mm}

\title{\huge Student: Tudor Andrei}
\\\vspace{10 mm}
\title{\huge Computers and Information Technology (Engligh Language)}
\\\vspace{10 mm}
\title{\huge Group 1.3 B}
\\\vspace{10 mm}
\title{\huge 1st Year}
\date{}
\maketitle

\newpage
\section*{Introduction}
\vspace{20 mm}
Assume there exist n activities with each of them being represented by a start time si and finish time fi. Two activities i and j are said to be non-conflicting if si ≥ fj or sj ≥ fi. The activity selection problem consists in finding the maximal solution set (S) of non-conflicting activities, or more precisely there must exist no solution set S' such that |S'| > |S| in the case that multiple maximal solutions have equal sizes.
\\\vspace{10 mm}
The activity selection problem is notable in that using a greedy algorithm to find a solution will always result in an optimal solution. A pseudocode sketch of the iterative version of the algorithm and a proof of the optimality of its result are included below. 

\newpage
\end{center}
\section*{Problem statement}
Schedule several competing activities that require
exclusive use of a common resource, with a goal of selecting a maximum-size
set of mutually compatible activities. Given that an activity has a start time and a
finish time two activities are compatible if the intervals between theirs start and
finish times don not overlap. Implement two different algorithms that resolve this
problem.




\newpage
\section*{Pseudocode}

\begin{lstlisting}
Greedy-Iterative-Activity-Selector(A, s, f):

    Sort A by finish times stored in f'
  
    S = {A[1]} 
  
    k = 1
  
    n = A.length
  
    for i = 2 to n:
 
       if s[i] ≥ f[k]: 
 
           S = S U {A[i]}
 
           k = i
 
    return S
\end{lstlisting}
\begin{lstlisting}
Weighted-Activity-Selection(S):  // S = list of activities
  
  sort S by finish time
  
  opt[0] = 0
  
  for i = 1 to n:
  
        t = binary search to find activity with finish time <= start time for i
  
        opt[i] = MAX(opt[i-1], opt[t] + w(i))
  
        return opt[n]
\end{lstlisting}
\begin{lstlisting}

\end{lstlisting}
\begin{lstlisting}

\end{lstlisting}



\newpage
\section*{Application design}
\vspace{10 mm}
For Greedy-Iterative-Activity-Selector we have:
\\\vspace{3 mm}
Line 1: This algorithm is called Greedy-Iterative-Activity-Selector, because it is first of all a greedy algorithm, and then it is iterative. There's also a recursive version of this greedy algorithm.
\\A is an array containing the activities.
\\s is an array containing the start times of the activities in A. 
\\f is an array containing the finish times of the activities in A. 

\\\vspace{3 mm}
\\ Line 3: Sorts in increasing order of finish times the array of activities A by using the finish times stored in the array f. This operation can be done, using for example merge sort, heap sort, or quick sort algorithms.

\\\vspace{2 mm}
\\Line 5: Creates a set S to store the selected activities, and initialises it with the activity 
A [ 1 ] that has the earliest finish time.
\\\vspace{1 mm}
\\Line 6: Creates a variable k 
 that keeps track of the index of the last selected activity.
\\\vspace{3 mm}
\\Line 10: Starts iterating from the second element of that array A.
 up to its last element.
\\\vspace{2 mm}
Lines 11,12: If the start time s [ i ]  of the 
i t h activity A [ i ] is greater or equal to the finish time f [ k ] of the last selected activity A [ k ] , then A [ i ] is compatible to the selected activities in the set S, and thus it can be added to S.  

\\\vspace{2mm}
Line 13: The index of the last selected activity is updated to the just added activity A [ i ].
\\\vspace{2mm}


\\\vspace{3 mm}
\\Weighted activity selection problem:

The generalized version of the activity selection problem involves selecting an optimal set of non-overlapping activities such that the total weight is maximized. Unlike the unweighted version, there is no greedy solution to the weighted activity selection problem. However, a dynamic programming solution can readily be formed using the following approach:
Consider an optimal solution containing activity k. We now have non-overlapping activities on the left and right of k. We can recursively find solutions for these two sets because of optimal sub-structure. As we don't know k, we can try each of the activities. This approach leads to an O ( n 3 ) solution. This can be optimized further considering that for each set of activities in 
( i , j ) , we can find the optimal solution if we had known the solution for ( i , t ) , where t is the last non-overlapping interval with j in ( i , j ) . This yields an O ( n 2 ) solution. This can be further optimized considering the fact that we do not need to consider all ranges ( i , j ), but instead just ( 1 , j ). 

\\


\newpage
\section*{Source Code}
\begin{lstlisting}
//------------------------------------------------------------
#include<stdio.h>
#include<algorithm>
#include<vector>
using namespace std;

void printMaxActivities(int s[], int f[], int n)
{
    int i, j;
    vector<int> myvector (f, f+sizeof(f)-1);
    sort(myvector.begin(),myvector.end());
    printf ("Following activities are selected n: ");
    i = 0;
    printf("%d ", i);
    for (j = 1; j < n; j++)
    { if (s[j] >= myvector[i])
      {
          printf ("%d ", j);
          i = j;
      }
    }
}

int main()
{
    int startTime[] =  {1, 3, 0, 5, 8, 5};
    int finishTime[] =  {4, 2, 7, 6, 9, 9};
    int n = sizeof(startTime)/sizeof(startTime[0]);
    printMaxActivities(startTime,finishTime, n);
    return 0;
}


\end{lstlisting}
\begin{lstlisting}
//------------------------------------------------------------


  This implementation assumes that the activities are already sorted according to their finish time.
  Prints a maximum set of activities that can be done by a single person, one at a time.
  n   -  Total number of activities
  s[] -  An array that contains start time of all activities
  f[] -  An array that contains finish time of all activities

  The first activity always gets selected, then we consider rest of the activities.

  If this activity has start time greater than or equal to the finish time of previously selected activity, then select it.


\end{lstlisting}
\begin{lstlisting}
//-----------------------------------------------------------------

#include <stdio.h>
#include <stdlib.h>

struct task    //structuri
{
    int start;
    int finish;
    int duration;
} taskuri[20],taskAux;

void bubbleSort(struct task arr[], int n)
{
    int i, j;                                     //orodonare cu metoda bulelorr
    for (i = 0; i < n-1; i++)
        for (j = 0; j < n-i-1; j++)
            if (arr[j].duration > arr[j+1].duration)
            {
                taskAux.start = arr[j].start;             //aux=x;
                taskAux.finish=arr[j].finish;
                taskAux.duration=arr[j].duration;

                arr[j].start = arr[j+1].start;              //x=y;
                arr[j].finish = arr[j+1].finish;
                arr[j].duration = arr[j+1].duration;

                arr[j+1].start= taskAux.start;              //y=aux;
                arr[j+1].finish= taskAux.finish;
                arr[j+1].duration= taskAux.duration;
            }
            else
                if(arr[j].duration == arr[j+1].duration)
                {
                    if(arr[j].start > arr[j+1].start)
                    {
                        taskAux.start = arr[j].start;             //aux=x;
                        taskAux.finish=arr[j].finish;
                        taskAux.duration=arr[j].duration;

                        arr[j].start = arr[j+1].start;              //x=y;
                        arr[j].finish = arr[j+1].finish;
                        arr[j].duration = arr[j+1].duration;

                        arr[j+1].start= taskAux.start;              //y=aux;
                        arr[j+1].finish= taskAux.finish;
                        arr[j+1].duration= taskAux.duration;

                    }

                }


}

int main()
{

    FILE *myFile;
    myFile = fopen("in.txt", "r");
    int numberArray[1000];
    int i,j;
    int n;
    int t;
    printf("time=");
    scanf("%d",&t);

    if (myFile == NULL)   // daca avem cv de citit
    {
        printf("Error Reading File\n");
        exit (0);
    }

    fscanf(myFile, "%d,", &numberArray[0]);
    n=numberArray[0];
    j=0;
    for (i = 1; i <= n*2; i++)                                   // citire din fisier
    {
        fscanf(myFile, "%d,", &numberArray[i] );
        if(i%2!=0)
        {
            taskuri[j].start=numberArray[i];
        }
        if(i%2==0)
        {
            taskuri[j].finish=numberArray[i];
            j++;
        }
    }

    for (i = 0; i < n; i++)                    // cat dureaza un task
    {
        taskuri[i].duration=taskuri[i].finish-taskuri[i].start;
    }

    bubbleSort(taskuri,n);           // apelam functia bublesort ce ordoneaza in functie de durata si de start

    for (i = 0; i < n; i++)
    {
        printf("task_%d: start_%d, finish_%d , duration_%d\n\n", i,taskuri[i].start,taskuri[i].finish,taskuri[i].duration );
    }

    int k=1;
    int aux;
    aux=taskuri[0].finish;
            for(i=0;i<=n-2;i++){                //ordonare
                for(j=i+1;j<=n-1;j++){
                if(aux<=taskuri[j].start)
                {
                    k=k+taskuri[j].duration;

                    aux=taskuri[j].finish;
                       i++;
                }
            }

}
printf("k=%d",k);
    fclose(myFile);

    return 0;
}


\end{lstlisting}




\newpage
\section*{Conclusions}
\vspace{20 mm}
Working on this project, I have gained a new appreciation for these relatively simple data structures and their many uses in the information world.  
\\\vspace{20mm}
\section*{References}
\large 1)www.stackoverflow.com

\\\vspace{6mm}
\\
2)http://www.sharelatex.com


\end{document}
